%-*-coding: utf-8;-*-
\documentclass[italian,a4paper]{scrartcl}
\usepackage{amsmath,amssymb,amsthm,thmtools}
\usepackage{eucal,babel,a4}
\usepackage[nochapters,pdfspacing]{classicthesis}
\usepackage[utf8]{inputenc}
\usepackage{graphicx,caption,subcaption}
\usepackage{bussproofs}
\usepackage{xcolor}

\newcommand{\RR}{{\mathbb R}}
\newcommand{\NN}{{\mathbb N}}
\newcommand{\C}{{\mathcal C}}
\newcommand{\defeq}{=}
\DeclareMathOperator{\diag}{diag}


\def\Xint#1{\mathchoice
{\XXint\displaystyle\textstyle{#1}}%
{\XXint\textstyle\scriptstyle{#1}}%
{\XXint\scriptstyle\scriptscriptstyle{#1}}%
{\XXint\scriptscriptstyle\scriptscriptstyle{#1}}%
\!\int}
\def\XXint#1#2#3{{\setbox0=\hbox{$#1{#2#3}{\int}$ }
\vcenter{\hbox{$#2#3$ }}\kern-.6\wd0}}
\def\dashint{\Xint-}

\declaretheoremstyle[
spaceabove=6pt, spacebelow=6pt,
headfont=\normalfont\itshape,
notefont=\mdseries, notebraces={(}{)},
bodyfont=\normalfont,
postheadspace=1em,
qed=,
shaded={rulecolor=pink!30,rulewidth=1pt,bgcolor=pink!10}
]{mystyle}

\declaretheorem[numberwithin=section,name=Teorema]{theorem}
\declaretheorem[sibling=theorem,name=Lemma]{lemma}
\declaretheorem[style=mystyle,sibling=theorem,name=Esercizio]{exercise}
\declaretheorem[style=mystyle,sibling=theorem,name=Esempio]{example}


\title{Real Numbers}
\author{E. Paolini}
\date{14 settembre 2015}

\begin{document}
\maketitle

The set $\RR$ of \emph{real numbers} is a set which: contains two special points
$0$ and $1$, has an operation $+$ and a relation $<$ defined on it
such that the following properties are satisfied:

\begin{enumerate}
\item
$1>0$, if $x<y$ then not $y<x$;
\item
Given any $x\in \RR$ there exists $y\in \RR$ such that $x+y=0$;
\item
for all $x,y,z\in \RR$ one has
\[
 x+y = y+x, \qquad
 (x+y)+z = x+(y+z), \qquad
 x < y \implies  x+z < y+z
\]
\item
if $A$ and $B$ are non empty subsets of $\RR$ satisfying $A\le B$ then there exists
$z\in \RR$ such that $A \le z$ and $z \le B$
(when we put a set in place of a point in any relation we mean that the relation
is valid for each point in the set).
\end{enumerate}

\section{natural numbers}

We say that a subset $A\subset \RR$ is \emph{inductive} if it satisfies these properties:
\begin{enumerate}
\item $0\in A$
\item $x\in A \Rightarrow x+1 \in A$.
\end{enumerate}
Define $\NN$ to be the intersection of all inductive subsets of $\RR$.

Of course $\NN$ is itself inductive and the following principle is valid
(check it!):

\emph{Inductive principle.} If $A$ is a subset of $\NN$ such that
\begin{enumerate}
\item $0\in A$;
\item $n\in A \Rightarrow n+1 \in A$
\end{enumerate}
then $A=\NN$.

Notice also that elements in $\NN$ are non-negative because non-negative numbers
are an inductive subset of $\RR$. Moreover if $n\in \NN$ then
$n+1 \neq 0$ because $n+1 > n$.

We hence have recovered Peano axioms of natural numbers:
\begin{enumerate}
\item $0 \in \NN$;
\item $n \in \NN \Rightarrow n+1 \in \NN$;
\item for all $n,m \in \NN$ if $n+1 = m+1$ then $n=m$;
\item for all $n\in \NN$ one has $n+1 \neq 0$;
\item if $A\subset \NN$ is inductive then $A= \NN$.
\end{enumerate}
\end{document}

Multiplication by
